\documentclass[10pt]{article}
\usepackage[utf8]{inputenc}
\usepackage{listings}
\usepackage{float}
\usepackage{graphicx}
\usepackage{fullpage}
\usepackage{caption}
\usepackage{subcaption}
\usepackage{amsmath}
\usepackage{hyperref}

%\renewcommand{\thesubsection}{\arabic{subsection}}
\renewcommand{\thesubsubsection}{\alph{subsubsection}}

\title{Pattern Recognition Practical 4}
\author{Group 24: \and Maikel Withagen (s1867733) \and Steven Bosch (s1861948)}
\date{\today}
\lstset{
frame=single, 
numbers=left, 
breaklines=true, 
language=Matlab,
basicstyle=\small, 
title=\lstname
}

\renewcommand{\thesection}{Assignment \arabic{section}}
\renewcommand{\thesubsection}{\arabic{subsection}}
\begin{document}
\section{Assignment 1}
\subsection{}
Using the code given in the appendix we created the scatterplot in figure \ref{fig1.1}.

\begin{figure}[H]
 \centering
 \includegraphics[width=\textwidth]{Fig1_1.png}
 \caption{Scatterplot for the two classes.}
 \label{fig1.1}
\end{figure}

The plot shows that there are at least three prototypes needed to approach a fairly well classification of these data. Two for set A, which should probably be located around $(3,4.5)$ and $(7.5, 5.5)$, and one for set B somewhere around $(5,5)$. 

\subsection{}
The code in the appendix shows our implementation of the LVQ1 algorithm. We acquired the following results for the different settings.

\subsubsection{1 Prototype for class A and 1 prototype for class B}

\begin{figure}[H]
 \centering
 \includegraphics[width=\textwidth]{Fig12_a1.png}
 \caption{Scatterplot for the two classes.}
 \label{fig1.12a1}
\end{figure}
\begin{figure}[H]
 \centering
 \includegraphics[width=\textwidth]{Fig12_a2.png}
 \caption{Training error rate to number of epochs.}
 \label{fig1.12a2}
\end{figure}

As is expected with only one prototype per class, the prototype for class A is formed quite well, allowing it to correctly classify at least the data points that belong to class A. However, since class B is distributed in two groups, with A in between them, the prototype for class B is formed in the center of one of the two clusters, which means it can only correctly classify that cluster correctly. The other cluster will be incorrectly classified as class A.

\subsubsection{}

\section*{Appendix}
\lstinputlisting{../Code/Ass1_1.m}
\lstinputlisting{../Code/Ass1_2.m}


\maketitle
\end{document}
